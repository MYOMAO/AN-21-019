\clearpage

\section{Introduction}
\label{sec:introduction}
Relativistic heavy ion collisions allow the study of quantum chromodynamics (QCD) at high energy densities and temperatures. Under such conditions, a state in which quarks and gluons are the relevant degrees of freedom, the quark-gluon plasma (QGP)~\cite{QGP1,QGP2}, is formed~\cite{Busza:2018rrf} as predicted by lattice QCD calculations~\cite{Karsch:2003jg}.
Multiple probes are necessary for characterizing the properties of the QGP medium.
Among these, heavy quarks, which are abundantly produced at the CERN LHC, have the potential of providing novel insights into QCD calculations, serving as probes of the QGP~\cite{Andronic:2015wma,Dong:2019byy}.
As they traverse the QGP, these hard-scattered partons lose energy by means of elastic collisions and medium-induced gluon radiation~\cite{Eloss1,Baier:2000mf,Chatrchyan:2011sx,Aad:2010bu}.
The study of parton energy loss can provide insights into the energy density and diffusion properties of the QGP.
The full reconstruction of beauty and charm gives access to their four-momenta and allows the study of the flavor and mass dependences of such processes.

In particular, beauty quarks are consider as a golden hard probes to study the transport properties of QGP. Beauty quarks are pre produced in the early stage of heavy-ion collisions. They retain their identities and traverse through the QGP before hadronization and decay. They record the entire evolution of the QGP. From the studies of the production cross sections and nuclear modification factor of b hadrons cross section in pp and heavy-ion collisions, we can understand the energy loss mechanism of beauty quarks in the QGP medium. From the ratios of the production yield of $\Lambda_b$ and $B^0_s$ to $B^+$ from low multiplicity pp, to PbPb, we can understand the beauty hadronization mechanism from small to large systems and test the QCD factorization theorem. These, along with the studies of charm and light flavor hadrons, will allow us to understand the flavor dependence of energy loss and probe the microscopic structure of QGP.


In this analysis, we will perform the measurements of $B^0_s$ and $B^+$ cross section as functions of transverse momentum, rapidity, and event multiplicity and within the CMS acceptance $\abs{y} < 2.4$ over a broad range in $pp$ collisions at $\sqrt{s_{NN}} = 5.02$ TeV with the CMS detector. We will use the 2017 pp datasets corresponding to an integrated luminosity of 302.3 $pb^{-1}$. We will follow the procedures of CADI HIN-17-008 and HIN-19-011 for the analysis. The decay channel we choose for $B^+$ and $B^0_s$ measurements are $B^0_s \to J/\psi \phi \rightarrow \mu^{+}\mu^{-}K^{+} K^{-}$ and $B^0_s \to J/\psi }K^{+} \to \mu^{+}\mu^{-}K^{+}$. We will use the 2018 PbPb $B^0_s$ and $B^+$ measurements to obtain the nuclear modification factor and $B^0_s/B^+$ ratios in pp and PbPb collisions. Our results will be important to help interpret other heavy flavor experimental results and constrain theoretical model calculations.  


\clearpage

